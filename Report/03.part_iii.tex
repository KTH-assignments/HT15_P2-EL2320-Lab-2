\subsection{Question 15}

The two main parameters that have a direct impact on the outlier detection approach are the set threshold $\lambda_{\Psi}$ and the 
measurement noise $Q$. As $|Q| \rightarrow 0$, the confidence in the correctness of the measurements increases, and the level of tolerance for discrepancies
between the prediction of measurements and their true values decreases. This means that more and more measurements are classified as outliers. Furthermore, the same 
result is observed for higher values of $\lambda_{\Psi}$. The higher its value is, the higher we demand the likelihood a particular measurement to be, and the more
probable it is that measurements are to be discarded as outliers.

\subsection{Question 16}

If no outliers are discarded, then all measurements are considered to be valid. However, in the general case, not all measurements are to be equally trusted.
Since even erroneous measurements will be taken into account, the weights of the particles are going to be erroneously distributed and the estimation of the
true state will be more incorrect than it would have been otherwise.
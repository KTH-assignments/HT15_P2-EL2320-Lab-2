\subsection{Question 1}

Particles are called the samples from a posterior distribution.

\subsection{Question 2}

An \textit{importance weight} $w_t^{[m]}$ is the probability of a measurement $z_t$ regarding a specific particle $x_t^{[m]}$: $w_t^{[m]} = p(z_t | x_t^{[m]})$.

In the context of particle filters, \textit{target distribution} is called the belief $bel(x_t)$, 
while \textit{proposal distribution} is called the $\overline{bel}(x_t)$. 

The target distribution cannot be sampled directly, however, it can be sampled indirectly by sampling from the proposal distribution. Each sample
has a probability of being drawn equal to its importance weight. 

\subsection{Question 3}

Even though a small number of particles results in a higher probability of particle deprivation, the main cause of \textit{particle deprivation} 
is the variance introduced by random sampling. The danger behind particle deprivation is that there may be no particles near the true state of the system.

\subsection{Question 4}

The resampling step is necessary in order to estimate the target distribution: it forces the particles to be distributed according to the posterior $bel(x_t)$.
If, on the other hand, no resampling took place, more particles would reside in areas with a low posterior probability, and would thus be wasted to 
regions of no real interest.

\subsection{Question 5}

In the case of a multimodal posterior, the average of the particle set would fail to account for the multiple regions of true interest, i.e. the regions where
the true state of the system has the highest probability of being.

\subsection{Question 6}

Inferences about states that lie between particles can be made using histograms, where a bin is comprised by a collection of adjacent particles, 
or Gaussian kernels, where each particle is used as the center of a Gaussian kernel.

\subsection{Question 7}

High sample variance can cause an incorrect estimation about the true state of a system. One solution is to reduce the resampling frequency. The second one uses a sequential stochastic process instead of drawing samples independently from each other.

\subsection{Question 8}

Pose uncertainty and number of particles used are proportionally related. The higher the uncertainty about the true state of the system, the larger the spread of 
the posterior distribution, hence the more particles are needed to estimate the potentially different modes of the distribution.

\subsection{Question 1}

In 2D space, equation $6$ uses the exact same value for the angle $\theta_0$ for all timesteps $t$. This equation would accurately estimate
the line of travel if no noise was introduced to the predicted state. However, because noise is present, the estimate of the motion will result in a zig-zag pattern,
unlike the one in 3D space. In 3D space, instead of a fixed angle, the control model uses the angle of travel from the previous timestep, which was infected by
noise, hence its estimate of the motion can be smoother than in the previous case.

\subsection{Question 2}

Since the initial translational and angular velocities $u_o, \omega_0$ need to be set in advance, the only circular motions that equation $9$ can model 
are instances of a specific circle, with a specific direction of travel.

\subsection{Question 3}

The constant part in equation $10$ serves for normalization purposes. The integral over the entire distribution needs to equal to $1.0$, since the distribution
represents a probability distribution.

\subsection{Question 4}

Algorithm $2$ illustrates the need for $M$ random numbers to be generated for the Multinomial re-sampling, where $M$ is the number of particles used.
In contrast, algorithm $3$ shows that only one random number needs to be generated in the context of Systematic re-sampling.

\subsection{Question 5}

In Multinomial re-sampling, a particle $p^{[k]}$ with an importance weight $w^{[k]}$ attached to it has a probability of being selected equal to $w^{[k]}$, 
and a probability of not being selected equal to $1 - w^{[k]}$. The probability of it not being selected $N$ consecutive times, since re-sampling takes place with replacement, is $(1 - w^{[k]})^N$. Hence, the probability of it being selected $N$ consecutive times is $1 - (1 - w^{[k]})^N$.

In Systematic re-sampling, a particle $p^{[k]}$ with a weight $w^{[k]} = M^{-1} + \epsilon, 0 < \epsilon << 1$ attached to it will always be selected since
the selection step size $M^{-1} < M^{-1} + \epsilon$. However, if $0 \leq w^{[k]} < M^{-1}$, the probability of its survival is $w^{[k]} \cdot M$. Since
the random number $r$ is selected in the interval $[0, M^{-1}]$, assuming a uniform distribution for the its selection, and given the fact that the probability of
selection of a particle is proportional to its weight, the probability of its selection is $\dfrac{w^{[k]} - 0}{M^{-1} - 0} = w^{[k]} \cdot M$.

\subsection{Question 6}

Measurement noise is modelled by $Sigma\_Q$, while process noise is modelled by $Sigma\_R$.

\subsection{Question 7}

If the process noise is set to zero, then the population of particles ends up being formed by $M$ copies of the same particle. This is reasonable, since
the process noise is essentially a factor which contributes to particle diversity.

\subsection{Question 8}

If no re-sampling is performed, then the population of particles in not transformed in any way. The consistency of the populations stays the same throughout
the experiment, and is comprised by the random particle set initialized at $t=1$. The motion of each particle simulates the estimation of the true state of the system
from a different initial state and no convergence is made.

\subsection{Question 9}

Low values for the standard deviation of the measurement noise ($Q(1,1) = Q(2,2) \sim 0.0001$) mean that we expect the estimate of the measurements to be very accurate. 
If these estimates do not follow the true measurements closely, then they are regarded as outliers. The end result is that the estimate does not converge to 
the true state. Increasing the standard deviation to values $Q(1,1) = Q(2,2) \sim 10000$ means that our uncertainty about the measurements increases, and so does
the variance of the particles around the true state.

\subsection{Question 10}

Low values for the standard deviation of the process noise ($Q(1,1) = Q(2,2) \sim 0.0001$) decrease the diversity of the particle population. This is exhibited
in the formation of a dense particle cloud which oscillates around the true state, with convergence not being guaranteed. As the standard deviation increases,
so does the probability of particles gaining more weight because of the larger variance in their position, which makes them estimate the true state quicker. The end
result is a particle population spread evenly around the true state. The radius of the particle cloud is proportionally related to the variance of the process noise.

\subsection{Question 11}

If the motion model models the true motion accurately, then the process noise can be kept low, since there is enough confidence in the estimate of the true motion.
On the other hand, if the motion model is known to be inaccurate, then the process noise should be larger, in order for particles to be able to estimate the
true motion.

\subsection{Question 12}

A motion model that describes the actual motion in an accurate manner will have higher precision and accuracy. Furthermore, fewer particles will be needed, since
the overall accuracy of the particles will be high. In contrast, an inaccurate motion model needs a larger particle population, so that the system can converge to 
the actual state.

\subsection{Question 13}

When there is a large discrepancy between a measurement and its prediction, the likelihood function takes a small value. Hence, we could install a threshold
to monitor this discrepancy and discard what we assume to be outliers.

\subsection{Question 14}

The following three sections illustrate the best settings for the covariance matrices $R, Q$.
As expected, the fixed motion model required the injection of higher levels of process noise in order for the particles to be able to
follow the true state, and was the most sensitive to it.
The linear and circular models were comparable, although the former required more measurement noise in order to be close to the latter.
Furthermore, the latter's best approximation of the true motion allowed for overall lower levels of noise. 

	\subsubsection{Fixed Motion}
	
	Adjusting the covariance matrices $R,Q$ as follows, it is possible to minimize the error to a value of $11.7 \pm 5.4$.
	
		\[
	R =
	\begin{bmatrix}
	    \mathbf{10} 		& 0		& 0 \\
	    0	    & \mathbf{10} 	& 0 \\
	    0     	& 0 	& 0.01 \\
	
	\end{bmatrix}
	\]
	
	
	\[
	Q =
	\begin{bmatrix}
	    200  & 0 \\
	    0    & 200 \\
	
	\end{bmatrix}
	\]
	
	
	\subsubsection{Linear Motion}
	
	Adjusting the covariance matrices $R,Q$ as follows, it is possible to minimize the error to a value of $7.1 \pm 3.8$.
	
		\[
	R =
	\begin{bmatrix}
	    \mathbf{3}       & 0		& 0 \\
	    0       & \mathbf{3} 	& 0 \\
	    0       & 0 	& 0.01 \\
	
	\end{bmatrix}
	\]
	
	
	\[
	Q =
	\begin{bmatrix}
	    400  & 0 \\
	    0    & 400 \\
	
	\end{bmatrix}
	\]

	\subsubsection{Circular Motion}
	
	Adjusting the covariance matrices $R,Q$ as follows, it is possible to minimize the error to a value of $6.8 \pm 3.7$.
	
		\[
	R =
	\begin{bmatrix}
	    \mathbf{2}  		& 0		& 0 \\
	    0       & \mathbf{2} 	& 0 \\
	    0       & 0 	& \mathbf{0.005} \\
	
	\end{bmatrix}
	\]
	
	
	\[
	Q =
	\begin{bmatrix}
	    200  & 0 \\
	    0    & 200 \\
	
	\end{bmatrix}
	\]